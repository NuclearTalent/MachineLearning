\documentclass[prc,amsart,english]{revtex4}
\usepackage{graphicx}
\usepackage{epsfig}
\usepackage{bm}
\usepackage{color}
\usepackage{float}
\usepackage{dcolumn}
\usepackage{multirow} 
\usepackage{hyperref}

\usepackage[T1]{fontenc}       % DC-fonts
\begin{document}
\title{Nuclear Talent Course on Machine Learning and Data Analysis for Nuclear Physics, ECT*, June 22-July 3, 2020}
\maketitle

\section{Introduction to the Talent Courses}

A recently established initiative, Training in Advanced Low Energy
Nuclear Theory, aims at providing an advanced and comprehensive
training to graduate students and young researchers in low-energy
nuclear theory. The initiative is a multinational network between
several European and Northern American institutions and aims at
developing a broad curriculum that will provide the platform for a
cutting-edge theory for understanding nuclei and nuclear
reactions. These objectives will be met by offering series of
lectures, commissioned from experienced teachers in nuclear
theory. The educational material generated under this program will be
collected in the form of WEB-based courses, textbooks, and a variety
of modern educational resources. No such all-encompassing material is
available at present; its development will allow dispersed university
groups to profit from the best expertise available.



The present document aims at giving a summary, 
with background information as well, of  the Nuclear Talent course, {\em Machine Learning and Data Analysis for Nuclear Physics}.
The course was held online via the  premises of the European Center for Theoretical Studies
in Nuclear Physics and Related Areas (ECT*), Trento, Italy during the period June 22 to July 3, 2020. 

This was the sixth course which has been held at the ECT*, the first one in 2012, then 2014, 2015, 2017, 2019 and finally this one in 2020. Although we had to run this course as a fully online course due to the global COVID-19 pandemic, the support from the ECT* has simply been crucial for the successful arrangement and running of these courses.
All practical arrangements, from accepting applications to the practical set up of the online course, were simply central for the success of the present  course held in the summer of 2020.
In total we had 157 participants from countries in Africa, Asia, Europe. Oceania and the Americas. The daily attendance varied from 110 to close to 150.




\section{Report }

\subsection{Introduction}
Probability theory and statistical methods play a central role in
science. Nowadays we are surrounded by huge amounts of data. For
example, there are about one trillion web pages; more than one hour of
video is uploaded to YouTube every second, amounting to 10 years of
content every day; the genomes of 1000s of people, each of which has a
length of more than a billion base pairs, have been sequenced by
various labs and so on. This deluge of data calls for automated
methods of data analysis, which is exactly what machine learning
provides. The purpose of this Nuclear Talent course was to provide an
introduction to the core concepts and tools of machine learning in a
manner easily understood and intuitive to physicists and nuclear
physicists in particular. We started with some of the basic methods
from supervised learning and statistical data analysis, such as
various regression methods before we moved into deep learning methods
for both supervised and unsupervised learning, with an emphasis on the
analysis of nuclear physics experiments and theoretical nuclear
physics. The students worked on hands-on daily examples as well as
projects that resulted in final credits.  The major scope was to give the
participants a deeper understanding on what Machine learning and Data
Analysis are and how they can be used to analyze data from nuclear
physics experiments and perform theoretical calculations of nuclear
many-body systems.


\subsection{Aims and outcomes}

The two-week online TALENT course on nuclear theory focused on Machine
Learning and Data Analysis algorithms for nuclear physics and how to
use such methods in the interpretation of data on the structure of
nuclear systems.

We proposed approximately twenty hours of lectures over two weeks and
a comparable amount of practical computer and exercise sessions,
including the setting of individual problems and the organization of
various individual projects.

The mornings consisted of lectures and the afternoons were devoted to
exercises meant to shed light on the exposed theory, the computational
projects and individual student projects. These components will be
coordinated to foster student engagement, maximize learning and create
lasting value for the students. For the benefit of the TALENT series
and of the community, material (courses, slides, problems and
solutions, reports on students' projects) have  been made publicly
available using version control software like git and posted
electronically on github (this site).



At the end of the course the students should have obtained a basic understanding of

\begin{itemize}
\item Statistical data analysis, theory and tools to handle large data sets.\item A solid understanding of central machine learning algorithms for supervised and unsupervised learning, involving linear and logistic regression, support vector machines, decision trees and random forests, neural networks and deep learning (convolutional neural networks, recursive neural networks etc)
\item Be able to write codes for linear regression, logistic regression and use modern libraries like Tensorflow, Pytorch, Scikit-Learn in order to analyze data from nuclear physics experiments and perform theoretical calculations
\item A deeper understanding of the statistical properties of the various methods, from the bias-variance tradeoff to resampling techniques.
\end{itemize}

We targeted an audience of graduate students (both Master of Science
and PhD) as well as post-doctoral researchers in nuclear experiment
and theory and due to the online format we had in total 157 participants.

The teaching team consisted of both theorists and experimentalists. We
believe such a mix is important as it gives the participants a better
understanding on how data are obtained, and what are the limitations
and possibilities in understanding and interpreting the experimental
information.


\subsection{Course Schedule}
  
Lectures were approximately 45 min each with a small break between each lecture. 
The schedule is included here, with links to online material as well.

\subsubsection{Week 1}

\begin{itemize}
\item  Monday June 22:  Linear Regression and intro to statistical data analysis	(Morten Hjorth-Jensen MHJ). Learning slides at
\url{https://nucleartalent.github.io/MachineLearningECT/doc/pub/Introduction/html/Introduction.html} and \url{https://nucleartalent.github.io/MachineLearningECT/doc/pub/Day1/html/Day1.html} and link to  video from lecture June 22 \url{https://mediaspace.msu.edu/media/t/1_ogq38oqq}
\item Tuesday June 23:	Logistic Regression and classification problems, intro to gradient methods	(MHJ). Learning slides at \url{https://nucleartalent.github.io/MachineLearningECT/doc/pub/Day2/html/Day2.html} and link to  video for first lecture at \url{https://mediaspace.msu.edu/media/t/1_po1a5e9v} and second lecture at \url{https://mediaspace.msu.edu/media/t/1_wbz4v2gm}
\item Wednesday June 24:	Decision Trees, Random Forests and Boosting methods (MHJ). Learning slides at \url{https://nucleartalent.github.io/MachineLearningECT/doc/pub/Day3/html/Day3.html} and link to  video at \url{https://mediaspace.msu.edu/media/t/1_vrt5rxls}
\item Thursday June 25:	Basics of Neural Networks and writing your own Neural Network code (MHJ). Learning slides at \url{https://nucleartalent.github.io/MachineLearningECT/doc/pub/Day4/html/Day4.html} and link to  video at \url{https://mediaspace.msu.edu/media/t/1_ksuz0ero}. The link to the video of the additional exercise session is at  \url{https://mediaspace.msu.edu/media/t/1_shte4iw5}
\item Friday June 26:	Beta-decay experiments, how to analyze various events, with hands-on examples . (Sean Liddick) Videos and teaching material \url{https://nucleartalent.github.io/MachineLearningECT/doc/pub/Day5/html/Day5.html}. Link to  video of online lecture at \url{https://mediaspace.msu.edu/media/1_5n2bssbl}. The link to the video of the additional exercise session is at \url{https://mediaspace.msu.edu/media/1_q74f31cw}
\end{itemize}

\subsubsection{Week 2}

\begin{itemize}
\item  Monday June 29:	Neural Networks and Deep Learning (Raghu Ramanujan, RR). PDF file of the presented slides at \url{https://nucleartalent.github.io/MachineLearningECT/doc/pub/Day6/pdf/Day6.pdf}. Jupter-Notebook at \url{https://nucleartalent.github.io/MachineLearningECT/doc/pub/Day6/ipynb/Day6.ipynb}. Video of lecture at \url{https://mediaspace.msu.edu/media/t/1_58a9xrbt}. Video of exercise session at \url{https://mediaspace.msu.edu/media/t/1_ulont3rg}
\item Tuesday June 30:	From Neural Networks to Convolutional Neural Networks and how to analyze experiment (classification of events and real data)	(Michelle Kuchera, MK). Jupyter-notebook of lecture at \url{https://nucleartalent.github.io/MachineLearningECT/doc/pub/Day7/ipynb/Day7.ipynb}. Video of lecture \url{https://mediaspace.msu.edu/media/t/1_2ysd5plh} and video of exercise at \url{https://mediaspace.msu.edu/media/t/1_watjxppf}
\item Wednesday July 1:	 Discussion of nuclear experiments and how to analyze data, presentation of simulated data from Active-Target Time-Projection Chamber (AT-TPC)	(Daniel Bazin). Slides of lectures (PDF) at \url{https://nucleartalent.github.io/MachineLearningECT/doc/pub/Day8/pdf/Day8.pdf}. Videos and teaching material \url{https://nucleartalent.github.io/MachineLearningECT/doc/pub/Day8/html/Day8.html}. Video of actual lecture at \url{https://mediaspace.msu.edu/media/t/1_azaquoc0}. Video of analysis of data with CNNs (MK) at \url{https://mediaspace.msu.edu/media/t/1_rozywc7h}. Jupyter-notebook of hands-on session at \url{https://nucleartalent.github.io/MachineLearningECT/doc/pub/Day8/ipynb/Day8.ipynb}
\item Thursday July 2: 	Generative models (MK). Slides of lectures (PDF) at \url{https://nucleartalent.github.io/MachineLearningECT/doc/pub/Day9/pdf/Day9.pdf}.
Jupyter-notebook at \url{https://nucleartalent.github.io/MachineLearningECT/doc/pub/Day9/ipynb/Day9.ipynb}.
Video of lecture at \url{https://mediaspace.msu.edu/media/t/1_ayfst99b}. Video of exercise session at \url{https://mediaspace.msu.edu/media/t/1_wpdmt7cw}. 
\item Friday July 3: 	Reinforcement Learning (RR). Slides of lectures (PDF) at \url{https://nucleartalent.github.io/MachineLearningECT/doc/pub/Day10/pdf/Day10.pdf}. Future directions in machine learning and summary of course. Video of first lecture at \url{https://mediaspace.msu.edu/media/t/1_0eiikln6}. Video of second lecture at \url{https://mediaspace.msu.edu/media/t/1_wzyabacr}. 
\end{itemize}





\subsection{Teaching}

The course was taught as an online intensive course of duration of two
weeks, with a total time of 20 h of lectures and 10 h of exercises,
questions and answers. Videos and digital learning material were made
available one week before the course begins. It was possible to work
on a final assignment of 2 weeks of work. The total load
wasapproximately 80 hours, corresponding to 5 ECTS in Europe. The
final assignment were graded with marks A, B, C, D, E and failed for
Master students and passed/not passed for PhD students. A course
certificate was issued for students requiring it from the University
of Trento.


The organization of a typical course day was as follows:

\subsubsection{Time and Activity}
\begin{itemize}
\item  2pm-4pm	(Central European time=CET) Lectures, project relevant information and directed exercises
\item 5pm-6pm	(CET) Questions and answers, Computational projects, exercises and hands-on sessions
\end{itemize}

\subsection{Teachers and organizers}
The teachers and organizers were
\begin{enumerate}
\item Daniel Bazin at Department of Physics and Astronomy and National Superconducting Cyclotron Laboratory, Michigan State University, East Lansing, Michigan, USA (DB)
\item Morten Hjorth-Jensen at Department of Physics and Astronomy and National Superconducting Cyclotron Laboratory, Michigan State University, East Lansing, Michigan, USA (MHJ)
\item Michelle Kuchera at Physics Department, Davidson College, Davidson, North Carolina, USA (MK)
\item Sean Liddick at Department of Chemistry and National Superconducting Cyclotron Laboratory, Michigan State University, East Lansing, Michigan, USA (SL)
\item Raghuram Ramanujan at Department of Mathematics and Computer Science, Davidson College, Davidson, North Carolina, USA (RR)
\end{enumerate}
Morten Hjorth-Jensen functioned as student advisor and coordinator.






\section{Brief summary}

The teachers were fully self-sponsored, and due to the COVID-19
situation worldwide there were no travel or logding expenses. The course was
fully online.

The support from the ECT*, and its year-long experiences with running doctoral training programs and Talent courses, was central to the success of the course. Of uttermost importance was Barbara Gazzoli's help will all administrative matters.  Without her help and the other staff of the ECT*, and the director's (Prof.~Jochen Wambach) 
enthusiastic support,  it is unlikely that this course would have run as smoothly as it did, or at all! 


Overall, we as teaching team feel the original learning outcomes and goals were met. It was a marvelous group of students to teach. Videos of the lectures will be published by Springer in due time.

This time we did however not perform a course survey, feedback from students is thus missing.

\end{document}












